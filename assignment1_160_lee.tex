% Options for packages loaded elsewhere
\PassOptionsToPackage{unicode}{hyperref}
\PassOptionsToPackage{hyphens}{url}
%
\documentclass[
]{article}
\usepackage{amsmath,amssymb}
\usepackage{lmodern}
\usepackage{ifxetex,ifluatex}
\ifnum 0\ifxetex 1\fi\ifluatex 1\fi=0 % if pdftex
  \usepackage[T1]{fontenc}
  \usepackage[utf8]{inputenc}
  \usepackage{textcomp} % provide euro and other symbols
\else % if luatex or xetex
  \usepackage{unicode-math}
  \defaultfontfeatures{Scale=MatchLowercase}
  \defaultfontfeatures[\rmfamily]{Ligatures=TeX,Scale=1}
\fi
% Use upquote if available, for straight quotes in verbatim environments
\IfFileExists{upquote.sty}{\usepackage{upquote}}{}
\IfFileExists{microtype.sty}{% use microtype if available
  \usepackage[]{microtype}
  \UseMicrotypeSet[protrusion]{basicmath} % disable protrusion for tt fonts
}{}
\makeatletter
\@ifundefined{KOMAClassName}{% if non-KOMA class
  \IfFileExists{parskip.sty}{%
    \usepackage{parskip}
  }{% else
    \setlength{\parindent}{0pt}
    \setlength{\parskip}{6pt plus 2pt minus 1pt}}
}{% if KOMA class
  \KOMAoptions{parskip=half}}
\makeatother
\usepackage{xcolor}
\IfFileExists{xurl.sty}{\usepackage{xurl}}{} % add URL line breaks if available
\IfFileExists{bookmark.sty}{\usepackage{bookmark}}{\usepackage{hyperref}}
\hypersetup{
  pdftitle={Review of decreasing cell adhesion protein during tumor progression},
  hidelinks,
  pdfcreator={LaTeX via pandoc}}
\urlstyle{same} % disable monospaced font for URLs
\usepackage[margin=1in]{geometry}
\usepackage{color}
\usepackage{fancyvrb}
\newcommand{\VerbBar}{|}
\newcommand{\VERB}{\Verb[commandchars=\\\{\}]}
\DefineVerbatimEnvironment{Highlighting}{Verbatim}{commandchars=\\\{\}}
% Add ',fontsize=\small' for more characters per line
\usepackage{framed}
\definecolor{shadecolor}{RGB}{248,248,248}
\newenvironment{Shaded}{\begin{snugshade}}{\end{snugshade}}
\newcommand{\AlertTok}[1]{\textcolor[rgb]{0.94,0.16,0.16}{#1}}
\newcommand{\AnnotationTok}[1]{\textcolor[rgb]{0.56,0.35,0.01}{\textbf{\textit{#1}}}}
\newcommand{\AttributeTok}[1]{\textcolor[rgb]{0.77,0.63,0.00}{#1}}
\newcommand{\BaseNTok}[1]{\textcolor[rgb]{0.00,0.00,0.81}{#1}}
\newcommand{\BuiltInTok}[1]{#1}
\newcommand{\CharTok}[1]{\textcolor[rgb]{0.31,0.60,0.02}{#1}}
\newcommand{\CommentTok}[1]{\textcolor[rgb]{0.56,0.35,0.01}{\textit{#1}}}
\newcommand{\CommentVarTok}[1]{\textcolor[rgb]{0.56,0.35,0.01}{\textbf{\textit{#1}}}}
\newcommand{\ConstantTok}[1]{\textcolor[rgb]{0.00,0.00,0.00}{#1}}
\newcommand{\ControlFlowTok}[1]{\textcolor[rgb]{0.13,0.29,0.53}{\textbf{#1}}}
\newcommand{\DataTypeTok}[1]{\textcolor[rgb]{0.13,0.29,0.53}{#1}}
\newcommand{\DecValTok}[1]{\textcolor[rgb]{0.00,0.00,0.81}{#1}}
\newcommand{\DocumentationTok}[1]{\textcolor[rgb]{0.56,0.35,0.01}{\textbf{\textit{#1}}}}
\newcommand{\ErrorTok}[1]{\textcolor[rgb]{0.64,0.00,0.00}{\textbf{#1}}}
\newcommand{\ExtensionTok}[1]{#1}
\newcommand{\FloatTok}[1]{\textcolor[rgb]{0.00,0.00,0.81}{#1}}
\newcommand{\FunctionTok}[1]{\textcolor[rgb]{0.00,0.00,0.00}{#1}}
\newcommand{\ImportTok}[1]{#1}
\newcommand{\InformationTok}[1]{\textcolor[rgb]{0.56,0.35,0.01}{\textbf{\textit{#1}}}}
\newcommand{\KeywordTok}[1]{\textcolor[rgb]{0.13,0.29,0.53}{\textbf{#1}}}
\newcommand{\NormalTok}[1]{#1}
\newcommand{\OperatorTok}[1]{\textcolor[rgb]{0.81,0.36,0.00}{\textbf{#1}}}
\newcommand{\OtherTok}[1]{\textcolor[rgb]{0.56,0.35,0.01}{#1}}
\newcommand{\PreprocessorTok}[1]{\textcolor[rgb]{0.56,0.35,0.01}{\textit{#1}}}
\newcommand{\RegionMarkerTok}[1]{#1}
\newcommand{\SpecialCharTok}[1]{\textcolor[rgb]{0.00,0.00,0.00}{#1}}
\newcommand{\SpecialStringTok}[1]{\textcolor[rgb]{0.31,0.60,0.02}{#1}}
\newcommand{\StringTok}[1]{\textcolor[rgb]{0.31,0.60,0.02}{#1}}
\newcommand{\VariableTok}[1]{\textcolor[rgb]{0.00,0.00,0.00}{#1}}
\newcommand{\VerbatimStringTok}[1]{\textcolor[rgb]{0.31,0.60,0.02}{#1}}
\newcommand{\WarningTok}[1]{\textcolor[rgb]{0.56,0.35,0.01}{\textbf{\textit{#1}}}}
\usepackage{graphicx}
\makeatletter
\def\maxwidth{\ifdim\Gin@nat@width>\linewidth\linewidth\else\Gin@nat@width\fi}
\def\maxheight{\ifdim\Gin@nat@height>\textheight\textheight\else\Gin@nat@height\fi}
\makeatother
% Scale images if necessary, so that they will not overflow the page
% margins by default, and it is still possible to overwrite the defaults
% using explicit options in \includegraphics[width, height, ...]{}
\setkeys{Gin}{width=\maxwidth,height=\maxheight,keepaspectratio}
% Set default figure placement to htbp
\makeatletter
\def\fps@figure{htbp}
\makeatother
\setlength{\emergencystretch}{3em} % prevent overfull lines
\providecommand{\tightlist}{%
  \setlength{\itemsep}{0pt}\setlength{\parskip}{0pt}}
\setcounter{secnumdepth}{-\maxdimen} % remove section numbering
\ifluatex
  \usepackage{selnolig}  % disable illegal ligatures
\fi

\title{Review of decreasing cell adhesion protein during tumor
progression}
\author{}
\date{\vspace{-2.5em}}

\begin{document}
\maketitle

\#Proteogenomics of Non-smoking Lung Cancer in East Asia Delinates
Molecular Signatures of Pathogenesis and Progression

\hypertarget{introduction}{%
\subsection{1.Introduction}\label{introduction}}

This study are collected cohort from in Taiwan, representing early
stage, predominantly female, non-smoking lung adenocarcinoma and
provided clinical feature for Taiwan cohort. Each gene had distinct
regulation patterns during tumors progression. Especially, I focused on
cell adhesion gene and DNA replication gene. In Figure 2I, we can see
that cell adhesion protein is decreasing as the tumor stage
increases(blue circle).

\begin{Shaded}
\begin{Highlighting}[]
\FunctionTok{library}\NormalTok{(knitr)}
\NormalTok{knitr}\SpecialCharTok{::}\FunctionTok{include\_graphics}\NormalTok{(}\StringTok{"/Users/ddw20/Pictures/Saved Pictures/figure.jpg"}\NormalTok{)}
\end{Highlighting}
\end{Shaded}

\includegraphics[width=16.17in]{/Users/ddw20/Pictures/Saved Pictures/figure}
So I'll process supplementary table to figure out gene related with cell
adhesion pathway and plot adhesion protein level in LUAD compared with
DNA replication protein level. As a result, I'll check whether the
actual TW cohort data shows down-regulation of adhesion pathway protein
and consider the relationship between tumor and cell adhesion protein.

\hypertarget{data-import}{%
\subsection{2. Data import}\label{data-import}}

\begin{Shaded}
\begin{Highlighting}[]
\FunctionTok{library}\NormalTok{(readxl)}
\FunctionTok{library}\NormalTok{(tidyverse)}
\end{Highlighting}
\end{Shaded}

\begin{verbatim}
## -- Attaching packages --------------------------------------- tidyverse 1.3.1 --
\end{verbatim}

\begin{verbatim}
## v ggplot2 3.3.5     v purrr   0.3.4
## v tibble  3.1.4     v dplyr   1.0.7
## v tidyr   1.1.3     v stringr 1.4.0
## v readr   2.0.1     v forcats 0.5.1
\end{verbatim}

\begin{verbatim}
## -- Conflicts ------------------------------------------ tidyverse_conflicts() --
## x dplyr::filter() masks stats::filter()
## x dplyr::lag()    masks stats::lag()
\end{verbatim}

I need information about clinical features of patients, pathway of gene,
and mRNA,protein log2Tumor/NAT value.

\begin{Shaded}
\begin{Highlighting}[]
\NormalTok{my\_table1\_S5 }\OtherTok{=} \FunctionTok{read\_excel}\NormalTok{(}\StringTok{\textquotesingle{}C:/Users/ddw20/Documents/bsms222\_160\_lee/TableS1.xlsx\textquotesingle{}}\NormalTok{, }\AttributeTok{sheet=}\DecValTok{5}\NormalTok{, }\AttributeTok{na=}\StringTok{"NA"}\NormalTok{)}
\NormalTok{my\_table1\_S6 }\OtherTok{=} \FunctionTok{read\_excel}\NormalTok{(}\StringTok{\textquotesingle{}C:/Users/ddw20/Documents/bsms222\_160\_lee/TableS1.xlsx\textquotesingle{}}\NormalTok{, }\AttributeTok{sheet=}\DecValTok{6}\NormalTok{, }\AttributeTok{na=}\StringTok{"NA"}\NormalTok{)}
\NormalTok{my\_table3\_S7 }\OtherTok{=} \FunctionTok{read\_excel}\NormalTok{(}\StringTok{\textquotesingle{}C:/Users/ddw20/Documents/bsms222\_160\_lee/TableS3.xlsx\textquotesingle{}}\NormalTok{, }\AttributeTok{sheet=}\DecValTok{7}\NormalTok{, }\AttributeTok{na=}\StringTok{"NA"}\NormalTok{)}
\NormalTok{my\_table6\_S2 }\OtherTok{=} \FunctionTok{read\_excel}\NormalTok{(}\StringTok{\textquotesingle{}C:/Users/ddw20/Documents/bsms222\_160\_lee/TableS6.xlsx\textquotesingle{}}\NormalTok{, }\AttributeTok{sheet=}\DecValTok{2}\NormalTok{, }\AttributeTok{na=}\StringTok{"NA"}\NormalTok{)}
\end{Highlighting}
\end{Shaded}

\hypertarget{data-wrangling}{%
\subsection{3. Data wrangling}\label{data-wrangling}}

Gene and related KEGG pathway is in \texttt{my\_table3\_S7(Table\ S3F)}.
I want to know some genes involved in the adhesion pathway and also
filter genes related to DNA replication pathway. so i used
\texttt{filter} and \texttt{grepl}.

\begin{Shaded}
\begin{Highlighting}[]
\NormalTok{adhesion}\OtherTok{\textless{}{-}}\NormalTok{my\_table3\_S7 }\SpecialCharTok{\%\textgreater{}\%} \FunctionTok{filter}\NormalTok{(}\FunctionTok{grepl}\NormalTok{(}\StringTok{\textquotesingle{}adhesion\textquotesingle{}}\NormalTok{, }\StringTok{\textasciigrave{}}\AttributeTok{C: KEGG pathway name}\StringTok{\textasciigrave{}}\NormalTok{))}\SpecialCharTok{\%\textgreater{}\%}\FunctionTok{count}\NormalTok{(}\StringTok{\textasciigrave{}}\AttributeTok{T: Gene name}\StringTok{\textasciigrave{}}\NormalTok{)}\SpecialCharTok{\%\textgreater{}\%} \FunctionTok{pull}\NormalTok{(}\StringTok{\textasciigrave{}}\AttributeTok{T: Gene name}\StringTok{\textasciigrave{}}\NormalTok{)}
\NormalTok{replication}\OtherTok{\textless{}{-}}\NormalTok{my\_table3\_S7 }\SpecialCharTok{\%\textgreater{}\%} \FunctionTok{filter}\NormalTok{(}\FunctionTok{grepl}\NormalTok{(}\StringTok{\textquotesingle{}replication\textquotesingle{}}\NormalTok{, }\StringTok{\textasciigrave{}}\AttributeTok{C: KEGG pathway name}\StringTok{\textasciigrave{}}\NormalTok{))}\SpecialCharTok{\%\textgreater{}\%}\FunctionTok{count}\NormalTok{(}\StringTok{\textasciigrave{}}\AttributeTok{T: Gene name}\StringTok{\textasciigrave{}}\NormalTok{)}\SpecialCharTok{\%\textgreater{}\%} \FunctionTok{pull}\NormalTok{(}\StringTok{\textasciigrave{}}\AttributeTok{T: Gene name}\StringTok{\textasciigrave{}}\NormalTok{)}
\end{Highlighting}
\end{Shaded}

\texttt{my\_table6\_S2(Table\ S6A)} has clinical features per patients.
I converted row and column in my\_table6\_S2 to use \texttt{tidy\ data}.

\begin{Shaded}
\begin{Highlighting}[]
\NormalTok{my\_table6\_S22}\OtherTok{\textless{}{-}}\FunctionTok{as.data.frame}\NormalTok{(}\FunctionTok{t}\NormalTok{(my\_table6\_S2))}
\FunctionTok{colnames}\NormalTok{(my\_table6\_S22)}\OtherTok{\textless{}{-}}\NormalTok{my\_table6\_S22[}\DecValTok{1}\NormalTok{,]}
\NormalTok{my\_table6\_S22}\OtherTok{\textless{}{-}}\NormalTok{my\_table6\_S22[}\SpecialCharTok{{-}}\FunctionTok{c}\NormalTok{(}\DecValTok{1}\NormalTok{,}\DecValTok{2}\NormalTok{,}\DecValTok{3}\NormalTok{,}\DecValTok{4}\NormalTok{,}\DecValTok{5}\NormalTok{),]}
\NormalTok{my\_table6\_S22}\OtherTok{\textless{}{-}}\NormalTok{tibble}\SpecialCharTok{::}\FunctionTok{rownames\_to\_column}\NormalTok{(my\_table6\_S22, }\StringTok{"ID"}\NormalTok{)}
\end{Highlighting}
\end{Shaded}

In \texttt{my\_table1\_S6(Table\ S1E)}, there are so many genes and
patients. I need expression level of only adhesion, replication gene.

\begin{Shaded}
\begin{Highlighting}[]
\NormalTok{my\_table1\_S66}\OtherTok{\textless{}{-}}\NormalTok{my\_table1\_S6}\SpecialCharTok{\%\textgreater{}\%}\FunctionTok{filter}\NormalTok{(Gene }\SpecialCharTok{\%in\%} \FunctionTok{c}\NormalTok{(adhesion,replication))}\SpecialCharTok{\%\textgreater{}\%}\FunctionTok{select}\NormalTok{(Gene, my\_table6\_S22}\SpecialCharTok{$}\NormalTok{ID)}
\NormalTok{my\_table1\_S66}\OtherTok{\textless{}{-}}\FunctionTok{as.data.frame}\NormalTok{(}\FunctionTok{t}\NormalTok{(my\_table1\_S66))}
\FunctionTok{colnames}\NormalTok{(my\_table1\_S66)}\OtherTok{\textless{}{-}}\NormalTok{my\_table1\_S66[}\DecValTok{1}\NormalTok{,]}
\NormalTok{my\_table1\_S66}\OtherTok{\textless{}{-}}\NormalTok{tibble}\SpecialCharTok{::}\FunctionTok{rownames\_to\_column}\NormalTok{(my\_table1\_S66, }\StringTok{"ID"}\NormalTok{)}
\NormalTok{my\_table1\_S66}\OtherTok{\textless{}{-}}\NormalTok{my\_table1\_S66[}\SpecialCharTok{{-}}\DecValTok{1}\NormalTok{,]}
\end{Highlighting}
\end{Shaded}

Using \texttt{pivot\_longer}, i made tidy data and added new column
`pathway' that distinguish DNA replication and Adhesion.

\begin{Shaded}
\begin{Highlighting}[]
\NormalTok{my\_table1\_S666}\OtherTok{\textless{}{-}}\NormalTok{my\_table1\_S66}\SpecialCharTok{\%\textgreater{}\%}\FunctionTok{pivot\_longer}\NormalTok{(}\SpecialCharTok{!}\NormalTok{ID,}\AttributeTok{names\_to=}\StringTok{"Gene"}\NormalTok{,}\AttributeTok{values\_to=}\StringTok{"log2TN"}\NormalTok{)}\SpecialCharTok{\%\textgreater{}\%}\FunctionTok{mutate}\NormalTok{(}\AttributeTok{pathway=}\FunctionTok{ifelse}\NormalTok{(Gene}\SpecialCharTok{\%in\%}\FunctionTok{c}\NormalTok{(replication),}\StringTok{"DNAreplication"}\NormalTok{,}\StringTok{"Adhesion"}\NormalTok{))}
\end{Highlighting}
\end{Shaded}

\hypertarget{visualizing}{%
\subsection{4.Visualizing}\label{visualizing}}

First, I merged \texttt{my\_table1\_S666(Table\ S1E)} and
\texttt{my\_table6\_S22(Table\ S6A)} to make one data frame including
patient ID, gene, pathway, protein log2 Tumor/NAT value. As we know ,
LUAD in TW cohort is characterized by predominant female patients. so I
used only female patient information to make a plot. Through LUAD female
cohort we can see DNA replication gene(MCM3,MCM2,LIG1) is up-regulation
in tumor.It is natural because tumors proliferate abnormally. Then, how
about cell adhesion gene? most adhesion genes were down-regulated in
LUAD patients except AKT, GRB2, MAPK1, PAK gene. Despite all being
involved in the adhesion pathway,Why does this difference occur? AKT,
GRB2, MAK1, PAK participate in other pathway. AKT is in not only focal
adhesion but also EGFR signaling, autophagy, AMPLpathway etc. It can be
seen that other proteins also participate in many different pathways. we
can figure out through my\_table3\_S7 and paper. especially, in Figure
2F, AKT, GRB2, MAPK is downstream protein of NLSCL pathway. and they
cause antiapoptosis, proliferation..

\begin{Shaded}
\begin{Highlighting}[]
\FunctionTok{merge}\NormalTok{(my\_table1\_S666,my\_table6\_S22,}\AttributeTok{by=}\StringTok{\textquotesingle{}ID\textquotesingle{}}\NormalTok{)}\SpecialCharTok{\%\textgreater{}\%}
  \FunctionTok{filter}\NormalTok{(}\SpecialCharTok{!}\FunctionTok{is.na}\NormalTok{(log2TN))}\SpecialCharTok{\%\textgreater{}\%}
  \FunctionTok{filter}\NormalTok{(Gender}\SpecialCharTok{==}\StringTok{"female"}\NormalTok{)}\SpecialCharTok{\%\textgreater{}\%}
  \FunctionTok{select}\NormalTok{(ID,Gene,log2TN,pathway,AngiolymphaticInvasion)}\SpecialCharTok{\%\textgreater{}\%} 
  \FunctionTok{mutate}\NormalTok{(}\AttributeTok{log2TN=}\FunctionTok{as.numeric}\NormalTok{(log2TN))}\SpecialCharTok{\%\textgreater{}\%}
  \FunctionTok{mutate}\NormalTok{(}\AttributeTok{Gene=}\FunctionTok{reorder}\NormalTok{(Gene, }\SpecialCharTok{{-}}\NormalTok{log2TN)) }\SpecialCharTok{\%\textgreater{}\%}
  \FunctionTok{filter}\NormalTok{(}\SpecialCharTok{!}\FunctionTok{is.na}\NormalTok{(AngiolymphaticInvasion))}\SpecialCharTok{\%\textgreater{}\%} 
  \FunctionTok{ggplot}\NormalTok{(}\FunctionTok{aes}\NormalTok{(ID,Gene,}\AttributeTok{fill=}\NormalTok{log2TN))}\SpecialCharTok{+}
  \FunctionTok{geom\_tile}\NormalTok{()}\SpecialCharTok{+}
  \FunctionTok{scale\_fill\_gradient2}\NormalTok{(}\AttributeTok{high =}\StringTok{"red"}\NormalTok{,}\AttributeTok{mid=}\StringTok{"white"}\NormalTok{,}\AttributeTok{low=}\StringTok{"blue"}\NormalTok{)}\SpecialCharTok{+}
  \FunctionTok{theme}\NormalTok{(}\AttributeTok{axis.text.x =} \FunctionTok{element\_text}\NormalTok{(}\AttributeTok{angle =} \DecValTok{90}\NormalTok{,}\AttributeTok{vjust =} \FloatTok{0.5}\NormalTok{, }\AttributeTok{hjust=}\DecValTok{1}\NormalTok{, }\AttributeTok{size=}\DecValTok{8}\NormalTok{))}\SpecialCharTok{+}
  \FunctionTok{facet\_grid}\NormalTok{(pathway}\SpecialCharTok{\textasciitilde{}}\NormalTok{., }\AttributeTok{switch=}\StringTok{"both"}\NormalTok{, }\AttributeTok{scales =} \StringTok{"free\_y"}\NormalTok{, }\AttributeTok{space =} \StringTok{"free\_y"}\NormalTok{)}\SpecialCharTok{+}
  \FunctionTok{labs}\NormalTok{(}\AttributeTok{x=}\StringTok{"patient"}\NormalTok{,}
       \AttributeTok{y=}\StringTok{"DNA replication gene vs Adhesion gene"}\NormalTok{,}
       \AttributeTok{title=}\StringTok{"Figure1. Adhesion protein expression }\SpecialCharTok{\textbackslash{}n}\StringTok{ compared with DNA replication protein expression in female TW cohort"}\NormalTok{,}
       \AttributeTok{fill=}\StringTok{"protein }\SpecialCharTok{\textbackslash{}n}\StringTok{ log2 T/N"}\NormalTok{)}
\end{Highlighting}
\end{Shaded}

\includegraphics{assignment1_160_lee_files/figure-latex/unnamed-chunk-8-1.pdf}

I associateed angiolymphatic invasion with cell adhesion. Instead of
classifying into tumor stages, i used angiolymphatic invasion
information to compare only the presence or absence of metastasis with
adhesion. we can see that protein log2 T/N of down-regulated adhesion
genes in fist plot is decreasing when angiolymphatic invasion is
``yes''. median protein log2 T/N of boxplot is decreased. ( AKT, CRB2,
PAK still remain an exception; they also participate in other pathways.)
cells are connected to ECM ,each other cell by cell-cell adhesion
including adherens junctions and focal adhesions. AS adhesion genes are
down-regulated, tumor cells invade and migrate into blood and lymph
node.

\begin{Shaded}
\begin{Highlighting}[]
\FunctionTok{merge}\NormalTok{(my\_table1\_S666,my\_table6\_S22,}\AttributeTok{by=}\StringTok{\textquotesingle{}ID\textquotesingle{}}\NormalTok{)}\SpecialCharTok{\%\textgreater{}\%}
  \FunctionTok{filter}\NormalTok{(}\SpecialCharTok{!}\FunctionTok{is.na}\NormalTok{(log2TN))}\SpecialCharTok{\%\textgreater{}\%}
  \FunctionTok{filter}\NormalTok{(Gender}\SpecialCharTok{==}\StringTok{"female"}\NormalTok{)}\SpecialCharTok{\%\textgreater{}\%}
  \FunctionTok{select}\NormalTok{(ID,Gene,log2TN,pathway,AngiolymphaticInvasion)}\SpecialCharTok{\%\textgreater{}\%} 
  \FunctionTok{mutate}\NormalTok{(}\AttributeTok{log2TN=}\FunctionTok{as.numeric}\NormalTok{(log2TN))}\SpecialCharTok{\%\textgreater{}\%}
  \FunctionTok{filter}\NormalTok{(}\SpecialCharTok{!}\FunctionTok{is.na}\NormalTok{(AngiolymphaticInvasion))}\SpecialCharTok{\%\textgreater{}\%} 
  \FunctionTok{filter}\NormalTok{(pathway}\SpecialCharTok{==}\StringTok{"Adhesion"}\NormalTok{) }\SpecialCharTok{\%\textgreater{}\%} 
  \FunctionTok{mutate}\NormalTok{(}\AttributeTok{regulation=}\FunctionTok{ifelse}\NormalTok{(Gene}\SpecialCharTok{\%in\%} \FunctionTok{c}\NormalTok{(}\StringTok{"AKT2"}\NormalTok{,}\StringTok{"AKT3"}\NormalTok{,}\StringTok{"GRB2"}\NormalTok{,}\StringTok{"MAPK1"}\NormalTok{,}\StringTok{"PAK1"}\NormalTok{,}\StringTok{"PAK2"}\NormalTok{),}\StringTok{"up{-}regulation"}\NormalTok{,}\StringTok{"down{-}regulation"}\NormalTok{))}\SpecialCharTok{\%\textgreater{}\%}
  \FunctionTok{ggplot}\NormalTok{(}\FunctionTok{aes}\NormalTok{(Gene,log2TN,}\AttributeTok{color=}\NormalTok{AngiolymphaticInvasion))}\SpecialCharTok{+}
  \FunctionTok{geom\_boxplot}\NormalTok{()}\SpecialCharTok{+}
  \FunctionTok{theme}\NormalTok{(}\AttributeTok{axis.text.x =} \FunctionTok{element\_text}\NormalTok{(}\AttributeTok{angle =} \DecValTok{90}\NormalTok{,}\AttributeTok{vjust =} \FloatTok{0.5}\NormalTok{, }\AttributeTok{hjust=}\DecValTok{1}\NormalTok{, }\AttributeTok{size=}\DecValTok{8}\NormalTok{))}\SpecialCharTok{+}
  \FunctionTok{facet\_grid}\NormalTok{(}\SpecialCharTok{\textasciitilde{}}\NormalTok{regulation, }\AttributeTok{scales =} \StringTok{"free\_x"}\NormalTok{, }\AttributeTok{space =} \StringTok{"free\_x"}\NormalTok{)}\SpecialCharTok{+}
  \FunctionTok{labs}\NormalTok{(}\AttributeTok{x=}\StringTok{"Adhesion gene"}\NormalTok{,}
       \AttributeTok{y=}\StringTok{"protein log2 T/N"}\NormalTok{,}
       \AttributeTok{title=}\StringTok{"Figure2.Adhesion protein expression level based on Angiolymphatic Invasion "}\NormalTok{,}
       \AttributeTok{color=}\StringTok{"Angiolymphatic }\SpecialCharTok{\textbackslash{}n}\StringTok{ Invasion"}\NormalTok{)}
\end{Highlighting}
\end{Shaded}

\includegraphics{assignment1_160_lee_files/figure-latex/unnamed-chunk-9-1.pdf}

\hypertarget{discussion}{%
\subsection{5. Discussion}\label{discussion}}

Patients with Angiolymphatic Invasion have down-regulation adhesion gene
protein levels. Because cell migrations are depending on the cell
adhesion properties. Dysregulation of adhesion gene is associated with
many pathological states including Angiolymphatic Invasion, cancer
metastasis. In addition to the genes shown in the Figure, there are many
genes participating in cell to cell adhesion or communication. Also, the
genes in the Figure are not only involved in the adhesion pathway. There
are genes that is related with other pathway(eg.NSCLC pathway) as well
as adhesion pathway, so some genes are upregulated despite of their
adhesion pathway.In general, as the tumor stage increases, the adhesion
protein log2Tumor/NAT decreases.

\end{document}
